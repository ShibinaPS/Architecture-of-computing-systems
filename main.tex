\documentclass[a4paper,12pt]{article}
\usepackage{cmap}
\usepackage[utf8]{inputenc}
\usepackage[warn]{mathtext}
\usepackage{epsf,amsmath,amsfonts,amssymb,amsbsy}
\usepackage[mathscr]{eucal}
\usepackage[english, russian]{babel}
\title{
\begin{centering}
Отчёт
\end{centering}
}
\usepackage[left=2cm,right=2cm,top=2cm,bottom=2cm]{geometry}
\usepackage{graphicx}
\usepackage{indentfirst}
\usepackage{wrapfig}
\usepackage{float}
\usepackage{mathtext}
\usepackage{amsmath}
\usepackage{siunitx} % Required for alignment
\usepackage{subfigure}
\usepackage{multirow}
\usepackage{rotating}
\usepackage{afterpage}
\usepackage[T1,T2A]{fontenc}
\usepackage[russian]{babel}
\usepackage{caption}
\usepackage[arrowdel]{physics}
\usepackage{booktabs}
\graphicspath{{pictures/}}
\DeclareGraphicsExtensions{.pdf,.png,.jpg}
\usepackage{pgfplots}
\begin{document}
	\maketitle

\newline
1. Шибина Полина Сергеевна

\newline
2. Б01-208

\newline
3. D-Flip-Flop

\newline
4. shibina.ps@phystech.edu

\newline
5. D-Flip-Flop - это запоминающее устройство в цифровых схемах, которое может хранить один бит данных. У него только один вход, кроме тактового сигнала, - "DATA". Выходной cигнал совпадает со входным, но задерживается на один такт.

\newline
Данный триггер имеет только один основной ввод, следовательно, может обрабатывать только один бит данных одновременно. Этот бит данных хранится до следующего тактового сигнала.

\newline
D-Flip-Flop имеет две выходные структуры - фактический выходной сигнал (Q) и его дополнение(!Q), принимающее обратное значение. Основной выходной сигнал будет таким же, как и входной, но с задержкой на один такт.

\newline
Триггер D-типа имеет структуру "ведущий-ведомый" ("master-slave"). Это включает в себя два этапа триггерного переключения - "ведущий" $ $ реагирует на состояние входа при поступлении тактового импульса, в то время как "ведомый" $ $ реагирует на выходной сигнал ведущего устройства, когда тактовый импульс заканчивается.

\newline
Структура "ведущий-ведомый" позволяет D-триггеру изменять состояния во время передачи тактового сигнала, а не непрерывно. Это предотвращает непредсказуемое поведение при изменении данных.

\newline
Когда тактовый сигнал переходит с низкого уровня на высокий, состояние входного сигнала DATA фиксируется и передается на выход Q. Например, если DATA было "1" $ $ в момент перехода, "Q"$ $ станет "1"$ $, а "!Q"$ $ станет "0". Это состояние сохраняется до следующего положительного фронта тактового сигнала, независимо от изменений значения DATA за это время.

\newline
Триггеры D-типа можно рассматривать как "хранители" $ $ цифровых данных. Они временно сохраняют и передают данные по мере необходимости, позволяя другим компонентам системы функционировать без непрерывного ввода данных.

\newline
6. 
\begin{table}[H]
\centering
\begin{tabular}{|c|c|c|c|}
\hline
CLK & DATA & $Q_{n}$ & $Q_{n+1}$ \\ \hline
0   & D   &  Q  & Q \\ \hline
1   & D   &  Q  & Q    \\ \hline
0 $\rightarrow$ 1   & D & Q & D \\ \hline
1 $\rightarrow$ 0   & D & Q & Q \\ \hline
\end{tabular}
\caption{Таблица истинности для D-Flip-Flop}
\end{table}\\



\begin{figure}[H]
	\begin{center}
		\includegraphics[width=15cm]{1.png}
	\end{center}
	\caption{\textit{Тактовый вход: низкий (логический уровень 0), данные на вход: 0, выходные данные: Q = 0, !Q = 1}}
	\label{ust}
        \end{figure}

\begin{figure}[H]
	\begin{center}
		\includegraphics[width=15cm]{2.png}
	\end{center}
	\caption{\textit{Тактовый вход: низкий (логический уровень 0), данные на вход: 1, выходные данные: Q = 0, !Q = 1}}
	\label{ust}
        \end{figure}

\begin{figure}[H]
	\begin{center}
		\includegraphics[width=15cm]{3.png}
	\end{center}
	\caption{\textit{Тактовый вход: высокий (логический уровень 1), данные на вход: 0, выходные данные: Q = 0, !Q = 1}}
	\label{ust}
        \end{figure}

\begin{figure}[H]
	\begin{center}
		\includegraphics[width=15cm]{4.png}
	\end{center}
	\caption{\textit{Тактовый вход: высокий (логический уровень 1), данные на вход: 1, выходные данные: Q = 1, !Q = 0}}
	\label{ust}
        \end{figure}


\newline
7. Критический путь
\begin{figure}[H]
	\begin{center}
		\includegraphics[width=15cm]{Критический путь.png}
	\end{center}
	\caption{\textit{Критический путь}}
	\label{ust}
        \end{figure}

\newline
8. Подсчёт транзисторов (без учёта PROBE, PIN и CLK):

\newline
1) 8x NAND (NAND = AND + NOT), итого 16 транзисторов;

\newline
2) NOT: 2, то есть 4 транзистора;

\newline
В итоге получаем 20 транзисторов в схеме.


\end{document}
